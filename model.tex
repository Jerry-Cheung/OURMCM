%%
%% This is file `mcmthesis-demo.tex',
%% generated with the docstrip utility.
%%
%% The original source files were:
%%
%% mcmthesis.dtx  (with options: `demo')
%% !Mode:: "TeX:UTF-8"
%% -----------------------------------
%%
%% This is a generated file.
%%
%% Copyright (C)
%%     2010 -- 2015 by latexstudio
%%     2014 -- 2016 by Liam Huang
%%
%% This work may be distributed and/or modified under the
%% conditions of the LaTeX Project Public License, either version 1.3
%% of this license or (at your option) any later version.
%% The latest version of this license is in
%%   http://www.latex-project.org/lppl.txt
%% and version 1.3 or later is part of all distributions of LaTeX
%% version 2005/12/01 or later.
%%
%% This work has the LPPL maintenance status `maintained'.
%%
%% The Current Maintainer of this work is Liam Huang.
%%
\documentclass{mcmthesis}
\mcmsetup{CTeX = true,   % 使用 CTeX 套装时,设置为 true
        tcn = 0000, problem = B,
        sheet = true, titleinsheet = true, keywordsinsheet = true,
        titlepage = true, abstract = true}
\usepackage{palatino}
\usepackage{lipsum}
\title{The \LaTeX{} Template for MCM Version \MCMversion}
\author{\small \href{http://www.latexstudio.net/}
  {\includegraphics[width=7cm]{mcmthesis-logo}}}
\date{\today}
\begin{document}



\section{Introduction}
\section{Assumptions}
\section{Model 1: T}
\subsection{Simulation and discussion}
.
.
.
\\
We simulate a toll plaze with 8 tollbooths and 3 lanes.

Commen cars are with  length of 4-4.5m and width of 1.65-1.85m,
so a car will take up 2 cells.
According to the 1994 Green Book,a fit solution of
 the plaze would be a trapezoid with
168 meters of recovery zone length and 612 meters of departure zone.
The width of one tollbooth and toll island
is usually 5.5 meters and the width of each lane is 3.5-4 meter
We set the length of each cell equal to 2 meter: ${l}_{car}=0.5(m)$.
So the parameters are as follows:
\[
{l}_{car}=2\\
{w}_{car}=1\\
WB={W}_{b}B=3\times8=24\\
WL={W}_{l}L=2\times3=6\\
{L}_{r}=84\\
{L}_{d}=306\\
\]

Figure~\ref{fig:q-p} shows the relationship between current and density of cars in
road with different ${p}_{v}$. %{翻译}当流量取得最大值q_max时对应的临界密度{d}_{c},基本图
%被临界图划分为两部分,当流量小于临界密度dc时,车流为自由流.
%pv是随机减速概率,it varies between different drivers.figure shows that pv越小,对应的
%capacity 越大,dc也越大。在下面的章节里,我们主要采用了Pv=0.5来讨论,因为这时capacity约为
%veh/h/lane,与实际情况个较为吻合

\begin{figure}[h]
\small
\centering
%\includegraphics[width=12cm]{q-p-difPv.png}
\caption{The relationship between throughput and density with different ${p}_{v}$}
 \label{fig:q-p}

\end{figure}







\end{document}

%%
%% This work consists of these files mcmthesis.dtx,
%%                                   figures/ and
%%                                   code/,
%% and the derived files             mcmthesis.cls,
%%                                   mcmthesis-demo.tex,
%%                                   README,
%%                                   LICENSE,
%%                                   mcmthesis.pdf and
%%                                   mcmthesis-demo.pdf.
%%
%% End of file `mcmthesis-demo.tex'.
