\documentclass{mcmthesis}
\mcmsetup{CTeX = true,   % 使用 CTeX 套装时,设置为 true
        tcn = 0000, problem = B,
        sheet = true, titleinsheet = true, keywordsinsheet = true,
        titlepage = false, abstract = true}
\usepackage{palatino}
\usepackage{lipsum}
\title{The \LaTeX{} Template for MCM Version \MCMversion}
\author{\small \href{http://www.latexstudio.net/}
  {\includegraphics[width=7cm]{mcmthesis-logo}}}
\date{\today}
\begin{document}
\begin{abstract}
\lipsum[1]%写summary的地方
\begin{keywords}
keyword1; keyword2
\end{keywords}
\end{abstract}
\maketitle
\tableofcontents
\newpage

\section{Introduction}
\subsection{Background}

Lewis Mumford, a famous sociologist and literary critic,
once said in a metaphorical manner, ``Adding highway lanes
to deal with traffic congestion is like loosening your
belt to cure obesity.`` Fortunately, he did not experience
the worse congestion around today`s highway toll plaza.

Currently, with roaring number of vehicles, rising
construction costs and constrained available areas,
traffic jam becomes more and more serious but future
toll-plaza construction opportunities are limited to
improve this situation markedly. Figure 1 shows the
congestion in the toll plaza near Tappan Zee Bridge.

\begin{figure}[h]
\small
\centering
\includegraphics[width=10cm]{figure1}
\caption{Toll Plaza Congestion}\label{fig1}
\end{figure}

Subject to the constraints referred above, neither
increasing highway lanes nor building more tollbooths
seems practical enough to relieve traffic jam around a
toll plaza nowadays, particularly for some heavily-traveled
 roads such as the Garden State Parkway, New Jersey.
 Therefore, looking for some innovative design improvements
  on the geometric parameters of the extent toll plaza
  is an effective solution.


\subsection{Restatement of the Problem}
In this paper, we are required to explore if there is a
better-than-ever toll plaza model with specific shape,
size, and merging pattern. In this model, the prerequisite
is that vehicles fan in from $B$ tollbooth egress lanes down
to $L$ ($B\textgreater L$) lanes of traffic (i.e., the number of both
tollbooths and the lanes after merging are fixed). We aim
to construct a model that can optimize the arrangement
according to the following conditions.

\begin{itemize}
\item Enhance the capability of the accident prevention(A).
\item Maximize the throughput(T).
\item Minimize the cost of the land and road
construction(C).
\end{itemize}

Through our analysis, we determine if there are better
solutions than any toll plaza in common use. Afterwards,
the performance of our solution in light and heavy traffic
and other various situations along with corresponding
sensitivity analysis is discussed.

\subsection{Our Work}

\section{Assumptions}

\section{Notations}

\section{Model}
\subsection{Time Cost and Construction Cost}

\subsubsection{Model establishing}
The throughput and construct cost of a toll plaza are
two contradictory indexes. Usually, the more throughput
is wanted, the more construct cost is paid. In order to
seek an optimal scheme to balance them both, we plan
to establish an objective function in which the throughput
is related to money consumption. So we introduce a
variable called unit waiting time cost, $C_h (USD/h/veh)$.
It signifies that if a vehicle queues for one hour at
a toll plaza, it will cause $C_h$ dollors loss.
Briefly speaking, the overall cost of a toll plaza
is defined as an aggregate of time cost and construct
cost.

With regard to a toll plaza, suppose the average
congestion time is $h_g$ ($h$) and the average enter
flow when congesting is $Q_g (veh/hs)$, then the total
waiting time is,


$$h_d =\int_o^{h_g} {(Q_g-Q_{max})(h_g-\tau)}\,d\tau=0.5{h_g}^2(Q_g-Q_{max})$$

Where, $Q_{max}$ is maximal throughput of this toll plaza.

If the designed service time of the toll plaza is y years,
then time cost during y years is,

$$C_t=365\times24h_{d}yC_h=4380h_g^2C_h(Q_g-Q_c\)y$$

Because we only think about the design of the departure
zone when calculating the construct cost, the cost of
approach zone and tollbooths is not included. But such
omission does not influence our conclusion. Construction
can be calculated by a linear function concerning area,
that is,
$$C_c=S_r\times C_r+S_l\times C_l$$

Where,
% Table generated by Excel2LaTeX from sheet 'Sheet1'
\begin{table}[htbp]
  \centering
    \begin{tabular}{rr}
    \toprule
    $S_r$  & The area of the road \\
    $C_r$  & The cost of the road per unit area \\
    $S_l$  & The area of the occupied land  \\
    $C_l$  & The cost of the occupied land per unit area \\
    \bottomrule
    \end{tabular}%
  \label{tab:addlabel}%
\end{table}%

Our goal is to minimize the total cost $C_s$
$$C_s=C_t+C_c$$
This chapter mainly discusses the effects of throughput
and construct cost on the toll plaza design, i.e.,
accident prevention is not included as a major research o
bject. Instead, some basic security indicators are
constraints when seeking the minimum $C_s$.
In the next step, we are establishing more explicit
relationships between size, shape as well as merging
pattern, and cost along with throughput.


From our perspective, connecting all the considerations
with cost directly or indirectly and make cost our major
objective function is an explicit and effective plan.
Thus, all our models are established out of this thought.
In detail, we can determine an average waiting time by
calculating the throughput of the toll plaza. In this way,
we may then quantify the average waiting time as money
consumption with an introduction of a uniform ``waiting
time cost``. Our goal is to look for the minimum cost
(including time and construct cost) in the case of
satisfying basic security conditions. In other words,
the overall cost is our objective function and security
factors are constraints towards the objective function.
We can get better solutions by minimizing the overall
cost.


\subsection{CA Model}
\subsubsection{Introduction}
In order to discover the influence of size, shape and merging pattern, we
 propose a two-dimensional Cellular Automata (CA) model. Our model is based on
 the one-dimensional \emph{Nagel Schreckenberg} model, which was first presented in
 1992 and successfully demonstrated many features of the traffic flow.

 Compared with a one-dimensional CA model, a two-dimensional one is more complex
 but feasible enough to simulate the real traffic flow. Therefore, the results
 from a two dimensional CA model is relatively more accurate.  Basically,
 the CA model can be regarded as an effective method to simulate features of
 traffic jams by showing how interactions between nearby vehicles cause the
 deceleration.
\subsubsection{Assumptions}
%用项目列表
\begin{itemize}
\item We assume that all the drivers are selfish
and short-sighted. To be specific, drivers always
adopt measures to move to the Roads Leading to Exit
(RLE), wherever they are.
\item We assume that once a vehicle leaves the
exit of the toll plaza, it will accelerate, namely
no congestion outside the plaza.
\item We assume that both the choices
towards tollbooths and the coming time of
vehicles satisfies random distribution.
\end{itemize}
\subsubsection{Model Establishing}
 In our two-dimensional CA model, each cell is
 evenly distributed in the shape of square. So the
 toll plaza, or rather, the whole departure area
 consists of plenty of square cells.
An independent cell or an adjacent cell cluster
can denote both empty roads or vehicles according
to their corresponding sizes, such as length and width.
We can assign a certain speed to each cell-formed vehicle.
But each speed value must be an integer, ranging from 0
to an identified $v_{max}$.
In addition, time is also discretized. Each time step
is defined as the time that a car takes to travel past
the length of 10 cars at the speed of the restricted
value.During a step interval, vehicles are set to perform
the following actions sent by corresponding instructions
in order.
Another important rule is that vehicles always perform
their updated actions at the same time.

Several actions are further explained as follows:
\begin{itemize}
\item \textbf{Acceleration:}

It reflects a characteristic
that vehicles tends to travel as fast as possible. Here,
this action obeys a rule as:

if $v_t<v_{max}$, then $v \rightarrow min(v_{t+1},v_{max})$
\item \textbf{Deceleration:}

 This action guarantee no
collision with the vehicle ahead. It satisfies:
$$v_t \rightarrow min(v_t, dx)$$
\item \textbf{Random deceleration:}

It embodies behavioral discrepancy of drivers. The
introduction of random deceleration basically reflects
the overreaction in decelerative processes. In conclusion,
it is a key factor that causes congestion. Likewise,
it meets following principle:

If $v_t > 0$, then $v_t \rightarrow v_{t-1}$ with probability $p_v$
\item \textbf{Steering:}

 It describes the trends that divers are more willing to
 turn to the Roads Leading to Exit (RLE) if they are not
 on the RLE at this moment. The corresponding rule is:

If (y\textgreater $W_L$), then $v_y=1$ with probability $p_y$

\item \textbf{Lane changing:} When a vehicle gets stuck,
the driver is likely to convert his or her lane to a n
earby one (only the one closer to the RLE) if that lane
is empty. It is also set to satisty:

If $dy\textgreater1$ and $v=0$, then $vy=1$ with
probability $p_d$

\item \textbf{Horizontal velocity:}
$$v_x=\sqrt{v^2 - vy^2}$$

\item \textbf{Motion:}

 It indicates that the position of a vehicle is
 shifted by its speed $v_x$ and $v_y$, thus,
 $x\rightarrow x+dx, y \rightarrow y+dy$

\item \textbf{Incoming vehicles:}

 Each vehicle will travel from the start of a
 certain tollbooth with the probability of $p_{in}$.
 This probability can denote traffic density in some
 ways.

 Where,

 % Table generated by Excel2LaTeX from sheet 'Sheet1'
 \begin{table}[htbp]
   \centering
     \begin{tabular}{ll}
     \toprule
     $v_t$  & The speed of current moment \\
     $v_{max}$ & The maximum speed allowed \\
     $v_{t+1}$ & The speed of next moment \\
     $dx$    & The nearest distance between two nearby vehicles in their horizontal direction \\
     $dy$    & The nearest distance between two nearby vehicles in their vertical direction \\
     $p_v$  & The probability of randomlization \\
     $p_y$  & The probability of sheering \\
     $p_d$  & The probability of lane changing \\
     $p_{in}$ & The probability of incoming vehicles \\
     $v_x$  & The horizontal velocity \\
     $v_y$  & The vertical velocity \\
     $W_L$  & The exit width of the toll plaza\\
     \bottomrule
     \end{tabular}%
   \label{tab:addlabel}%
 \end{table}%



\end{itemize}
\subsubsection{Simulation and discussion}
We convert our thoughts and design above into program instructions via Python,
and simulate a toll plaze with 8 tollbooths and 3 lanes of travel.

Common vehicles are usually $4-4.5m$ long and $1.65-1.85m$ wide,
so a vehicle will take up 2 cells.
According to the \emph{Green Book, 1994}, an appropriate design of
a toll plaze would be a trapezoid with a
168-meter-long recovery zone and a 612-meter-long departure zone.
The width of a tollbooth along with a toll island
is usually $5.5m$, while that of each lane is $3.5-4m$.
We set the length of each cell equal to $2m$: $l_{car}=4m$.
So the parameters are as follows:
\[
\begin{align*}
&{l}_{veh}=2m\\
&{W}_{veh}=1m\\
&W_{B}={W}_{b}B=3\times8=24m\\
&W_{L}={W}_{l}L=2\times3=6m\\
&{L}_{r}=84m\\
&{L}_{d}=306m
\end{align*}
\]



Figure~\ref{fig:q-p} shows the relationship between current throughput and
traffic flow density with different ${p}_{v}$.

%{翻译}当流量取得最大值q_max时对应的临界密度{d}_{c},zhegetu
%被临界midu划分为两部分,当流量小于临界密度dc时,车流为自由流.
%pv是随机减速概率,it varies between different drivers.figure shows that pv越小,对应的
%capacity 越大,dc也越大。在下面的章节里,我们主要采用了Pv=0.5来讨论,因为这时capacity约为700
%veh/h/lane,与实际情况个较为吻合

\begin{figure}[h]
\small
\centering
\includegraphics[width=12cm]{q-p-difPv.png}
\caption{The relationship between throughput and density with different ${p}_{v}$}
\label{fig:q-p}

\end{figure}

The density in the correspondence with the maximal throughput $q_{max}$ is
defined as critical density $d_{c}$. We can distinguish that the curves in
Figure~\ref{fig:q-p} are divided into two sections by that critical density.
If the flow density is lower than $d_{c}$, it will be a free flow (as Figure shows);%注意图片对应
 otherwise, a crowded flow (as Figure shows).%没写是图几
%插入两张图片free_flow&crowded_flow
We can conclude from the simulations above that the RLE are the most crowded
part once traffic density increases to cause congestion. In this case, drivers
always consider turning to the RLE as soon as possible, which causes unevenly
distribution of traffic density in the departure zone.


As mentioned above, ${p}_{v}$ is the probability of random deceleration.
It varies among different drivers. It is not difficult to conclude
that the smaller the value is, the larger $q_{max}$ is accessible. At the same
time, $d_{c}$ will increase correspondingly. In the following chapters, we
choose to adopt ${p}_{v}=0.5$ as our basis, because the $q_{max}$ at this time is
is consistent with the actual value, approximately 700 $veh/h/lane$. And more
detailed simulation results will also be displayed.

\section{Size}

The size of the merge area can be determined by
the following parameters:
\begin{itemize}
\item Total width of typical toll lanes ($W_{B}$).
\item Length of the recovery zone ($L_{r}$).
\item Length of total departure zone($L_{d}$).
\item Width of the exit($W_{L}$).
\end{itemize}
Parameters hereinbefore are shown in Figure \ref{fig4}.
\begin{figure}[h]
\small
\centering
\includegraphics[width=10cm]{figure4}
\caption{The Parameters}\label{fig4}
\end{figure}
For the number of travel lanes ($L$) is fixed, $W_{L}$ is
constant. Then we are considering the effect of the
rest parameters separately. By simulating our model
mentioned above via computer program, we can figure
out how these parameters affect the maximal throughput
of the merge area, that is, $Q_{max}$.
Figure \ref{fig5} shows the variation tendency of
$Q_{max}$ with
the alteration of the width of each tollbooth $W_{b}$.
Apparently $W_{B} = B \times W_{b}$. Figure \ref{fig5}
provides a result
under the prerequisite that $W_b$ ranges from 6 to 14
while other parameters are fixed.
\begin{figure}[h]
\small
\centering
\includegraphics[width=10cm]{figure5}
\caption{The Linear Fitting Image of $Q_{max}$ and $W_b$}\label{fig5}
\end{figure}
We utilize an appropriate Linear Fitting Function Model
to address the data, and then get the fitting function
of $Q_{max}$ and $W_b$:
$$Q_{max}=p_{1}\times W_b+p_2$$
Where, $$p_1 =10.35, p_2 = 661.7$$
The simulation result indicates that $Q_{max}$ would only
be affected by the total width of typical toll lanes
($W_B$) in a small degree. However, increasing $W_b$ will
markedly result in a rise in construction costs.
For $L_r$, the linear fitting image is showed in Figure \ref{fig6}
and the variance of $Q_{max}$ is 36.7188. We can see that
$L_r$ causes almost no effect on the merge area capacity.
In the Linear Fitting Function, the coefficient $p_3=0$.
\begin{figure}[h]
\small
\centering
\includegraphics[width=10cm]{figure6}
\caption{The Linear Fitting Image of $Q_{max}$ and $L_r$}\label{fig6}
\end{figure}

Both linear fitting image, and function of $Q_{max}$ and
$L_d$ are shown below. There is a negative correlation
between $Q_{max}$ and $L_d$. Nevertheless, the relationship
is so faint that enlarging $Q_{max}$ by changing $L_d$ is
not functional.
\begin{figure}[h]
\small
\centering
\includegraphics[width=10cm]{figure7}
\caption{The Linear Fitting Image of $Q_{max}$ and $L_d$}\label{fig7}
\end{figure}

$${Q_{3max}} = {p_5}\times {L_d} + {p_6}$$

Where, $$p_5 =-0.1248, p_6 = 801.1.$$
From discussion above, the size does cause impact on
$Q_{max}$, while the impact is not that obvious.
In addition, $L_d$ and $W_B$ should never be constructed
too small because it may cause potential safety
problems and result in higher accident rate.
For ensuring safety, the departure taper rates $T_r$
 must be limited into $[T_{rmin}, T_{rmax}]$. In summary,
 in order to determine the optimal size of a toll
 plaza, the problem can be transformed into a linear
 minimization problem with the form:

 \begin{equation*}
 \begin{split}
  {\bf minimize}\quad &$ $C$ = $S$($C_{road}$ + {$C_{construct}$}) + 4380$h_g^2${C_h}({Q_g} - {Q_{\max }})$\\
   {\bf s.t}\quad  & $$Q_{max}$|W_b = 3m, L_d = 612m, L_r = 168m = 709XL\\
       & \frac{dQ_{max}}{dW_{b}}=p_{1}=10.35\\
       & \frac{dQ_{max}}{dL_{r}}=p_{3}=0\\
       & \frac{dQ_{max}}{dL_{d}}=p_{5}=-0.1248\\
       & T_{rmin}\textless T_r\textless T_{rmax}\\
       & W_b\textgreater W_{bmin}\\
 \end{split}
 \end{equation*}

Here $W_{bmin}$ signifies the minimal width of the toothbooths, and
the area of toll plaza
$$S=13W_b(L_r+0.5L_d)+0.5LW_{L}L_{d}$$
The departure taper rate
$$T_r=\frac{L_d}{13W_b-LW_{L}}

For example, there is a toll plaza with three lanes
and eight tollbooths. To solve the problem, we can
make assumptions as following:
\begin{itemize}
\item The limited speed is 30 km/h.
\item The lifespan planned reaches to 10 years.
\item The average daily congestion time $h_g = 1h$.
\item The average congestion flow $Q_q=2300 veh/h$.
\item The land price locally $C_{land}=85 USD/m^2$
\item The cost of highway construction $C_{road}=357 USD/m_2$
\end{itemize}

According to \emph{1994 Green Book} taper rate for lane
addition in a 3-lane section ,$T_r$ should arrange from 8
to 15. Commonly, it takes 1 USD as the cost for each per
son to wait one hour.


On the basis of these conditions, the optimal solution of
linear programming is
\begin{align}
             W_b&=5   \\
             L_r&=47m \\
             L_d&=265.5m
\end{align}
the total cost  $$C=8,167,645USD$$


\section{Shape}

We propose two types of the plaza shape:
series type and parallel type.
\subsection{Series Type}
Literally, this type is to connect two or more merge
areas in series. Here, we only consider connecting
two merge areas. Furthermore, we might as well suppose
$B=8$ and $L=3$. Specially, vehicles fan in from eight
tollbooth egress lanes down to six lanes of traffic,
then fan in from six lanes of traffic to three, as
Figure \ref{fig8} shows.
\begin{figure}[h]
\small
\centering
\includegraphics[width=10cm]{figure8}
\caption{Series Type}\label{fig8}
\end{figure}

According to the Buckets Effect (BE),
$$Q_{smax}=min⁡\left\{ Q_{amax},Q_{bmax} \right\}$$
$Q_{amax}$, $Q_{bmax}$ and $Q_{smax}$ respectively
signify the
maximal throughput of the primary merge area,
second stage merge area and the whole series-type
toll plaza.
We can get Table \ref{tab1} from simulation results, which
indicates the value of the maximal throughput for
each traffic line($Q_{emax}$) with different
$B$ and $L$ ($B\textgreater L$).
% Table generated by Excel2LaTeX from sheet 'Sheet1'
\begin{table}[htbp]
  \centering
  \caption{$Q_{max}$ with B and L}\label{tab1}
    \begin{tabular}{lllllllllll}
    \toprule
    $Q_{max}$     & B=1   & B=2   & B=3   & B=4   & B=5   & B=6   & B=7   & B=8   & B=9   & B=10 \\
    \midrule
    L=1   &       & 882   & 845   & 832   & 796   & 771   & 772   & 736   & 689   & 640 \\
    L=2   &       &       & 815   & 789   & 773   & 755   & 720   & 718   & 686   & 659 \\
    L=3   &       &       &       & 755   & 758   & 734   & 724   & 709   & 684   & 671 \\
    L=4   &       &       &       &       & 724   & 700   & 715   & 695   & 694   & 673 \\
    L=5   &       &       &       &       &       & 716   & 695   & 690   & 688   & 673 \\
    L=6   &       &       &       &       &       &       & 695   & 688   & 682   & 667 \\
    L=7   &       &       &       &       &       &       &       & 682   & 676   & 670 \\
    L=8   &       &       &       &       &       &       &       &       & 676   & 660 \\
    L=9   &       &       &       &       &       &       &       &       &       & 651 \\
    \bottomrule
    \end{tabular}%
  \label{tab:addlabel}%
\end{table}%

As for the example shown in Figure \ref{fig8},
$$Q_{smax}=min⁡\left\{ 688\times6,734\times3 \right\}$$
For a simple toll plaza with the same number of B and L,
$$Q_{max}=709\times3=2127$$
Therefore
$$Q_{smax}>Q_{max}$$
Moreover, since $Q_{emax}$ is becoming large as $B$ or $L$ d
ecrease, we can prove that the merge area in series
type would have a larger capacity for any $B$ and $L$ ($B\textmore L$).
Thus, connecting two or more merge area in series is a
practical and optimized scheme.


\subsection{Parallel Type}
That is, divide the merge area transversely and put
them together in parallel. Similarly, if we suppose
$B=8$ and $L=3$ again, the toll plaza can be divided into
two portions as Figure \ref{fig9} shows.
\begin{figure}[h]
\small
\centering
\includegraphics[width=10cm]{figure9}
\caption{Parallel Type}\label{fig9}
\end{figure}
Since the two areas are juxtaposed,
$$Q_{pmax}=Q_{mmax}+Q_{nmax}$$
$Q_{mmax}$, $Q_{nmax}$ and $Q_{pmax}$ respectively
signify the maximal throughput of the merge area m,
merge area n and the whole parallel-type toll plaza.
Similar to the analysis of the series type,
$Q_{emax}$ is becoming large with the increasing
of $B$ or $L$. Thus, this solution could enlarge
the maximal throughput efficaciously.

\subsection{An example}
For the convenience of discussion, we still take a
toll plaza with three lanes and eight tollbooths
as an example. If we adopt the method of 8-6-3 series
type, from what has been discussed above, $Q_{max}$
increases from 2127 to 2212 veh/h with increasing
rate of 3.9\%. Therefore, the cost time decreases
 by 1.4\%.


However, at the same time, the area $S_r$ and $S_d$
increases. The total area will increase by
$$\Delta S=4W_{b}L_{d}+W_{L}(6L_r+3L_d)$$
We can solve out that the total area will
increase by 73.4\%.Thus, the construction cost will
also increase by 73.4\%. Under normal conditions
, construction cost and time cost are of the
same order of magnitude, so the series cannot
solve the problem practically.

On the other hand, if we adopt the method
of parallel type ,we can find that Qmax
increase by 13.3\%, which is fully significant.
And we adopt the structure shown in figure \ref{fig10}
simultaneously. Such shape would cause the total
area to increase by the following formula:
$$\Delta S=6W_{L}(L_D+L_r+L_A)$$
\begin{figure}[h]
\small
\centering
\includegraphics[width=10cm]{figure10}
\caption{Parallel Type}\label{fig10}
\end{figure}
Where $L_{A}$ is the length of approving zone of
the toll plaza, which increases by 21.3\% approximately. Hence it is necessary to construct
toll plaza in this way when the traffic congestion
is serious and the cost time is extremely large.


\section{Merging Pattern}

Here, we devise a real-time merging control system for
toll plaza based on the previous work by M. Papageorgiou
et al. Through our improvement, it can be specially used
for the toll plaza we are discussing. In addition, this
system can effectively maximize the throughput by
maintaining the occupancy of departure area close to a
critical value. Figure \ref{fig2} illustrates the framework of
this system.

\begin{figure}[h]
\small
\centering
\includegraphics[width=15cm]{figure2}
\caption{The Framework of This System}\label{fig2}
\end{figure}
%此处的figure被放到了下一页



\textbf{Merge area}

As a matter of fact, the merge area is equal to the
departure zone as referred to above. Typically, it
is an approximately trapezoidal area where the vehicles
leave from the booths on a total of $B$ lanes and finally
fit into $L$ lanes of the exit. Here, we focus on the
flow-density variation with the occupancy increasing
in the merge area. Eventually,according to our CA
model's simulation, we obtain a diagram to
describe this functionary relationship, which is shown
in Figure \ref{fig3}.

\begin{figure}[h]
\small
\centering
\includegraphics[width=10cm]{figure3}
\caption{The Functionary Relationship}\label{fig3}
\end{figure}

After noticing that X-axis is occupancy $o$ (\%), while
Y-axis represents the exit flow $q_{out}$, we can tell from
the diagram:
\begin{itemize}
\item When $o$ is small, merging conflicts are scarce.
the exit flow is correspondingly low and it
 increases linearly with density increasing
\item As $o$ increases, merging conflicts may increase, but
$q_{out}$ also increases as well until, for a specific value
$o_{cr}$, the exit flow reaches the capacity $q_{cap}$.
\item If $o$ increases beyond $o_{cr}$, merging conflicts
become more frequent, leading to a serious congestion.
Consequently, a capacity drop happens.
\end{itemize}
Therefore, we can conclude that the occupancy of the merge
area can directly influence the exit flow, or rather, the
throughput. And we can regulate the occupancy under the
goal to maintain $o  \approx  o_{cr}$ by controlling the merging
pattern with the assistant of a control algorithm and
feedback. From a macroscopic perspective, the maximum
throughput can be achieved by a certain merging pattern
design. As a result, our goal is to model this design.

\textbf{Feedback control based on PID controller}

We are inspired by the commonly used PID control algorithm in industrial
control systems and
decide to deploy traffic lights to individual lanes
as control devices.


However, the most crucial task is to determine the
form of feedback control.

Our goal is to ensure that occupancy is maintained
at around $o_{cr}$ regardless of how serious the traffic congestion is.
What's different from common PID control system is that the traffic
system is not a Continuous system,and
The control loops are very long compared to other systems.
We suppose that the feedback control is activated
at each discrete time interval witch usually set as 1~2 min. After activation, it
will collect latest measurements of occupancy $o$, and
send data-converted instructions to control devices
under the purpose of maintaining $o  \approx  o_{cr}$.


The PID system can be expressed as:

  \[q\left( n \right) = K_{p}e\left( {n - 1} \right) + {K_i}\sum_{j=1}^{n-1} e\left( {n - 1} \right)+\frac{d}{dx} \left[ e\left( {n - 1} \right)-e\left( {n - 2} \right)\right]
  \]
  \[
  e\left( {n} \right)=\hat{o}\left( {n} \right)-o\left( {n} \right)
  \]
Where,

% Table generated by Excel2LaTeX from sheet 'Sheet1'
\begin{table}[htbp]
  \centering
    \begin{tabular}{ll}
      \toprule
    $n$     & The discrete time index \\
    $q(n)$  & The controlled entering flow (veh/h) to be implemented in a new time step $n$ \\

    $o(n)$ & The measured occupancy of merge area in this time step \\
    $\hat{o}$ & The desired value of occupancy (can be set as $o_{cr}$) \\
    $K_{p},K_i,K_d$  & coefficients for the proportional, integral, and derivative terms, always positive \\
    \bottomrule
    \end{tabular}%
  \label{tab:addlabel}%
\end{table}%

In addition, the occupancy measurement should best be
placed at or just upstream of the location where serious
vehicle decelerations (congestion) appear first.

\section{Conclusion}

\section{Sensitivity Analysis}
\subsection{The Performance of Our Solution in Light
and Heavy Traffic}
\subsection{Autonomous Vehicles}
\subsection{The Proportions of Different Tollbooths}

\section{Strengths and Weaknesses}
\subsection{Strengths}
\subsection{Weaknesses}

\begin{thebibliography}{99}
\bibitem{1} D. E. KNUTH   The \TeX{}book  the American
Mathematical Society and Addison-Wesley
Publishing Company , 1984-1986.
\bibitem{2}Lamport, Leslie,  \LaTeX{}: `` A Document Preparation System '',
Addison-Wesley Publishing Company, 1986.

\end{thebibliography}

\begin{appendices}

\section{First appendix}


Here are simulation programmes we used in our model as follow.\\

\textbf{\textcolor[rgb]{0.98,0.00,0.00}{Input matlab source:}}
\lstinputlisting[language=Matlab]{./code/mcmthesis-matlab1.m}

\section{Second appendix}

some more text \textcolor[rgb]{0.98,0.00,0.00}{\textbf{Input C++ source:}}
\lstinputlisting[language=C++]{./code/mcmthesis-sudoku.cpp}

\end{appendices}
\end{document}

%%
%% This work consists of these files mcmthesis.dtx,
%%                                   figures/ and
%%                                   code/,
%% and the derived files             mcmthesis.cls,
%%                                   mcmthesis-demo.tex,
%%                                   README,
%%                                   LICENSE,
%%                                   mcmthesis.pdf and
%%                                   mcmthesis-demo.pdf.
%%
%% End of file `mcmthesis-demo.tex'.
